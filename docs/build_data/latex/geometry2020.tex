%% Generated by Sphinx.
\def\sphinxdocclass{report}
\documentclass[letterpaper,10pt,english]{sphinxmanual}
\ifdefined\pdfpxdimen
   \let\sphinxpxdimen\pdfpxdimen\else\newdimen\sphinxpxdimen
\fi \sphinxpxdimen=.75bp\relax

\PassOptionsToPackage{warn}{textcomp}
\usepackage[utf8]{inputenc}
\ifdefined\DeclareUnicodeCharacter
% support both utf8 and utf8x syntaxes
  \ifdefined\DeclareUnicodeCharacterAsOptional
    \def\sphinxDUC#1{\DeclareUnicodeCharacter{"#1}}
  \else
    \let\sphinxDUC\DeclareUnicodeCharacter
  \fi
  \sphinxDUC{00A0}{\nobreakspace}
  \sphinxDUC{2500}{\sphinxunichar{2500}}
  \sphinxDUC{2502}{\sphinxunichar{2502}}
  \sphinxDUC{2514}{\sphinxunichar{2514}}
  \sphinxDUC{251C}{\sphinxunichar{251C}}
  \sphinxDUC{2572}{\textbackslash}
\fi
\usepackage{cmap}
\usepackage[T1]{fontenc}
\usepackage{amsmath,amssymb,amstext}
\usepackage{babel}



\usepackage{times}
\expandafter\ifx\csname T@LGR\endcsname\relax
\else
% LGR was declared as font encoding
  \substitutefont{LGR}{\rmdefault}{cmr}
  \substitutefont{LGR}{\sfdefault}{cmss}
  \substitutefont{LGR}{\ttdefault}{cmtt}
\fi
\expandafter\ifx\csname T@X2\endcsname\relax
  \expandafter\ifx\csname T@T2A\endcsname\relax
  \else
  % T2A was declared as font encoding
    \substitutefont{T2A}{\rmdefault}{cmr}
    \substitutefont{T2A}{\sfdefault}{cmss}
    \substitutefont{T2A}{\ttdefault}{cmtt}
  \fi
\else
% X2 was declared as font encoding
  \substitutefont{X2}{\rmdefault}{cmr}
  \substitutefont{X2}{\sfdefault}{cmss}
  \substitutefont{X2}{\ttdefault}{cmtt}
\fi


\usepackage[Bjarne]{fncychap}
\usepackage{sphinx}

\fvset{fontsize=\small}
\usepackage{geometry}


% Include hyperref last.
\usepackage{hyperref}
% Fix anchor placement for figures with captions.
\usepackage{hypcap}% it must be loaded after hyperref.
% Set up styles of URL: it should be placed after hyperref.
\urlstyle{same}

\addto\captionsenglish{\renewcommand{\contentsname}{Contents:}}

\usepackage{sphinxmessages}
\setcounter{tocdepth}{1}



\title{geometry2020}
\date{Dec 10, 2020}
\release{1.2}
\author{Gaetano Vitale}
\newcommand{\sphinxlogo}{\vbox{}}
\renewcommand{\releasename}{Release}
\makeindex
\begin{document}

\pagestyle{empty}
\sphinxmaketitle
\pagestyle{plain}
\sphinxtableofcontents
\pagestyle{normal}
\phantomsection\label{\detokenize{index::doc}}

\phantomsection\label{\detokenize{index:module-main}}\index{module@\spxentry{module}!main@\spxentry{main}}\index{main@\spxentry{main}!module@\spxentry{module}}\index{cartesian\_representation\_line() (in module main)@\spxentry{cartesian\_representation\_line()}\spxextra{in module main}}

\begin{fulllineitems}
\phantomsection\label{\detokenize{index:main.cartesian_representation_line}}\pysiglinewithargsret{\sphinxcode{\sphinxupquote{main.}}\sphinxbfcode{\sphinxupquote{cartesian\_representation\_line}}}{\emph{\DUrole{n}{a}}, \emph{\DUrole{n}{b}}, \emph{\DUrole{n}{type}\DUrole{o}{=}\DUrole{default_value}{1}}}{}
This function print the cartesian presentation of a line
a: numpy\sphinxhyphen{}array of the first point
b: numpy\sphinxhyphen{}array of the direction (type = 0) or of the second point (type = 1)

\end{fulllineitems}

\index{conic\_section\_classification() (in module main)@\spxentry{conic\_section\_classification()}\spxextra{in module main}}

\begin{fulllineitems}
\phantomsection\label{\detokenize{index:main.conic_section_classification}}\pysiglinewithargsret{\sphinxcode{\sphinxupquote{main.}}\sphinxbfcode{\sphinxupquote{conic\_section\_classification}}}{\emph{\DUrole{n}{coeff}\DUrole{o}{=}\DUrole{default_value}{{[}{]}}}}{}
This function provides a classification of a conic section

coeff: list of the coefficient of the equation of the conic section

if the equation is

A x\textasciicircum{}2 + B xy + C y\textasciicircum{}2 + D x + E y + F = 0

then the array coeff is

{[}A,B,C,D,E,F{]}

\end{fulllineitems}

\index{gauss\_elimination() (in module main)@\spxentry{gauss\_elimination()}\spxextra{in module main}}

\begin{fulllineitems}
\phantomsection\label{\detokenize{index:main.gauss_elimination}}\pysiglinewithargsret{\sphinxcode{\sphinxupquote{main.}}\sphinxbfcode{\sphinxupquote{gauss\_elimination}}}{\emph{\DUrole{n}{matrix}}}{}
This function compute  Gauss elimination process
matrix: numpy\sphinxhyphen{}array

\end{fulllineitems}

\index{linear\_dependence() (in module main)@\spxentry{linear\_dependence()}\spxextra{in module main}}

\begin{fulllineitems}
\phantomsection\label{\detokenize{index:main.linear_dependence}}\pysiglinewithargsret{\sphinxcode{\sphinxupquote{main.}}\sphinxbfcode{\sphinxupquote{linear\_dependence}}}{\emph{\DUrole{n}{A}}}{}
This function answer to the question “Are these vectors linearly independent?”

A : numpy\sphinxhyphen{}array matrix with vectors as rows

\end{fulllineitems}

\index{linear\_equations() (in module main)@\spxentry{linear\_equations()}\spxextra{in module main}}

\begin{fulllineitems}
\phantomsection\label{\detokenize{index:main.linear_equations}}\pysiglinewithargsret{\sphinxcode{\sphinxupquote{main.}}\sphinxbfcode{\sphinxupquote{linear\_equations}}}{\emph{\DUrole{n}{matrix}}, \emph{\DUrole{n}{vector}}}{{ $\rightarrow$ None}}
this function resolve a system of linear equations
:param matrix: matrix of coefficients
:param vector: vector of constant terms

\begin{sphinxVerbatim}[commandchars=\\\{\}]
\PYG{g+gp}{\PYGZgt{}\PYGZgt{}\PYGZgt{} }\PYG{n}{linear\PYGZus{}equations}\PYG{p}{(}\PYG{n}{np}\PYG{o}{.}\PYG{n}{eye}\PYG{p}{(}\PYG{l+m+mi}{2}\PYG{p}{)}\PYG{p}{,}\PYG{n}{array}\PYG{p}{(}\PYG{p}{[}\PYG{l+m+mi}{1}\PYG{p}{,}\PYG{l+m+mi}{1}\PYG{p}{]}\PYG{p}{)}\PYG{p}{)}
\PYG{g+go}{[1,1]}
\end{sphinxVerbatim}

\end{fulllineitems}



\chapter{Indices and tables}
\label{\detokenize{index:indices-and-tables}}\begin{itemize}
\item {} 
\DUrole{xref,std,std-ref}{genindex}

\item {} 
\DUrole{xref,std,std-ref}{modindex}

\item {} 
\DUrole{xref,std,std-ref}{search}

\end{itemize}


\renewcommand{\indexname}{Python Module Index}
\begin{sphinxtheindex}
\let\bigletter\sphinxstyleindexlettergroup
\bigletter{m}
\item\relax\sphinxstyleindexentry{main}\sphinxstyleindexpageref{index:\detokenize{module-main}}
\end{sphinxtheindex}

\renewcommand{\indexname}{Index}
\printindex
\end{document}